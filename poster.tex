\documentclass[a0,portrait]{a0poster}
% You might find the 'draft' option to a0 poster useful if you have
% lots of graphics, because they can take some time to process and
% display. (\documentclass[a0,draft]{a0poster})

% Switch off page numbers on a poster, obviously, and section numbers too.
\pagestyle{empty}
\setcounter{secnumdepth}{0}

% The textpos package is necessary to position textblocks at arbitary 
% places on the page.
\usepackage[absolute]{textpos}
\usepackage[percent]{overpic}
\usepackage{lipsum}
% Graphics to include graphics. Times is nice on posters, but you
% might want to switch it off and go for CMR fonts.
\usepackage{graphics,wrapfig,times}

% These colours are tried and tested for titles and headers. Don't
% over use color!
\usepackage{color}
\definecolor{DarkBlue}{rgb}{0.1,0.1,0.5}
\definecolor{Red}{rgb}{0.9,0.0,0.1}
% see documentation for a0poster class for the size options here
\let\Textsize\normalsize
\def\Head#1{\noindent\hbox to \hsize{\hfil{\LARGE\color{DarkBlue} #1}}\bigskip}
\def\LHead#1{\noindent{\LARGE #1}\smallskip}
\def\Subhead#1{\noindent{\large #1}}
\def\Title#1{\noindent{\VeryHuge\color{white} #1}}

% Set up the grid
%
% Note that [40mm,40mm] is the margin round the edge of the page --
% it is _not_ the grid size. That is always defined as 
% PAGE_WIDTH/HGRID and PAGE_HEIGHT/VGRID. In this case we use
% 15 x 25. This gives us a wide central column for text (7 grid
% spacings) and two narrow columns (3 each) at each side for 
% pictures, separated by 1 grid spacing.
%
% Note however that texblocks can be positioned fractionally as well,
% so really any convenient grid size can be used.
%
\TPGrid[40mm,40mm]{15}{25}  % 3 - 1 - 7 - 1 - 3 Columns

% Mess with these as you like
\parindent=0pt
%\parindent=1cm
\parskip=0.5\baselineskip

% abbreviations
\newcommand{\ddd}{\,\mathrm{d}}

\begin{document}
\begin{overpic}
{/home/killian/fyp_poster/SCSS_background.png}
\put (0,0) {\includegraphics{tcd/SCSS.png}}
\end{overpic}

\begin{overpic}[unit=1mm]
    {tcd/Title_Background.png}
 \put (3,1) {\Title{I Love Posters}}
\end{overpic}

%\includegraphics{tcd/Title_Background.png}

% Understanding textblocks is the key to being able to do a poster in
% LaTeX. In
%
%    \begin{textblock}{wid}(x,y)
%    ...
%    \end{textblock}
%
% the first argument gives the block width in units of the grid
% cells specified above in \TPGrid; the second gives the (x,y)
% position on the grid, with the y axis pointing down.

% You will have to do a lot of previewing to get everything in the 
% right place.

% This gives good title positioning for a portrait poster.
% Watch out for hyphenation in titles - LaTeX will do it
% but it looks awful.
%\begin{textblock}{12}(0,1.8)
%\baselineskip=3\baselineskip \Title{Anonymous Digital Cash using Zero Knowledge Proofs}
%  Format With a\\Second title line below}
%\end{textblock}
%\begin{textblock}{17}(0,2.5)
\begin{center}
\LHead{Student: Patrick Losty Supervisor: Arthur Guiness}
%\end{textblock}
\end{center}

% An example text block, to get you started!
\begin{textblock}{7}(0,3)
  \LHead{Introduction}
  \lipsum
 
\end{textblock}


% Another text block in the bottom right.
\begin{textblock}{7}(8,3)
  \LHead{Some stuff}
\lipsum  
\end{textblock}

% If you want to add a figure do something like this:

%\begin{textblock}{3}(0,15)
%  \begin{center}
%\resizebox{3\TPHorizModule}{!}{\includegraphics{my_figure.eps}}
%\\Figure 5: Googles per Snark (renormalised with wild angry men
%  \end{center}
%\end{textblock}
\begin{textblock}{7}(0,16)
\LHead{Things}

\lipsum[8]

\end{textblock}

\begin{textblock}{7}(8,16)
\LHead{Conclusions}

\lipsum[13]

\end{textblock}

% Place the group logo at the bottom left - visually this balances
% well with the University logo at the top right. 
\begin{textblock}{3}(0,24)
  \begin{center}
\resizebox{2.5\TPHorizModule}{!}{\includegraphics{/home/killian/fyp_poster/TCD_LOGO.png}}
  \end{center}
\end{textblock}

\end{document}

